\documentclass[a4paper,12pt]{jsreport}
\usepackage[dvipdfmx]{graphicx}
\usepackage{amssymb}
\usepackage{amsmath}
\usepackage{float}

%%%%%%document start%%%%%%

%%%%%%title page%%%%%%
\begin{document}
\begin{center}
\thispagestyle{empty}
{\Large 卒業論文}\\
\vspace*{1cm}
{\Huge 卒業論文 テンプレート}\\
\vspace*{2cm}
{\Large 関西学院大学 理工学部 情報科学科}\\
\vspace*{0.5cm}
{\LARGE 学籍番号 \ 名前}\\
\vspace*{9cm}
{\Large 2021年吉日}\\
\vspace*{2cm}
{\LARGE 指導教員 徳山豪 教授}
\end{center}
%%%%%%end title%%%%%%

%%%%%%目次ページ(いじらないでください)%%%%%%
\newpage
\pagenumbering{arabic}
\tableofcontents

%%%%start main%%%%%%

%%%%%%%%%%%%%%%%%%%%%%%%%%%%%%%%%%%%%%%%%%%%%%%%%%
\newpage
\chapter{はじめに}
目次を作る際は\verb+\tableofcontents+ と打ちます.\\
新しいページに区切るときは\verb+\newpage+ と打ちます.
\section{セクション}
\verb+\chapter+と打つことで第○章を作成することができます.\\
○.○は\verb+\section+と打ちましょう. 同様にしてサブセクション, サブサブセクションも作成可能です.

%%%%%%%%%%%%%%%%%%%%%%%%%%%%%%%%%%%%%%%%%%%%%%%%%%
\chapter{本文の書き方(数式や図の挿入)}
\section{セクション}
本文中に参考文献\cite{sample}のように挿入したいときは, \verb+\cite{著書名}+と打ちましょう.
以下のようにして, 数式を挿入します. ”TeXコマンド一覧”と検索することで, コマンドの詳細を知ることができます.
\begin{equation}
Y=ax^3+bx^2+cx+d \\
\end{equation}
\begin{eqnarray}
E[3^S] & = & \sum ^{N} _{k=0}{} _{N}\mathrm{C}_{k} \left( \frac{3}{9} \right)^{k} \left( \frac{8}{9} \right)^{N-k} \\
       & = & \left( \frac{11}{9}\right)^N \\
\end{eqnarray}
\subsubsection{サブサブセクション}
\begin{eqnarray}
\int^\frac{\pi}{2} _0 \sin^4\theta \cos^2 \theta {\rm d} \theta &=& \frac{1}{2} \cdot 2 \int^\frac{\pi}{2} _0 \sin^{2 \cdot \frac{5}{2} -1}\theta \cos^{2 \cdot \frac{3}{2}-1} \theta {\rm d}\theta\\
&=& \frac{1}{2} \cdot \rm B \left( \frac{5}{2} , \frac{3}{2} \right)\nonumber \\
&=& \frac{\Gamma \left(\frac{5}{2} \right)\Gamma \left(\frac{3}{2} \right)}{2\cdot\Gamma \left( 4 \right)}\\
&=& \frac{\frac{3}{2} \cdot \frac{1}{2} \cdot \sqrt{\pi} \cdot \frac{1}{2} \cdot \sqrt{\pi}}{2 \cdot 3! }  =  \frac{3 \cdot \pi}{2^5 \cdot 3}\nonumber \\
&=& \frac{\pi}{32}
\end{eqnarray}

%%%%%%%%%%%%%%%%%%%%%%%%%%%%%%%%%%%%%%%%%%%%%%%%%%
%%%%%%参考文献%%%%%%
%参考文献はgrad.bibというファイルに書き込みます.%
\bibliographystyle{junsrt}
\bibliography{grad.bib}

%%%%%%%%%%%%%%%%%%%%%%%%%%%%%%%%%%%%%%%%%%%%%%%%%%
\chapter*{付録}
\verb+\chapter*+と打つことで、第○章をタイトルのみにすることができます.\\
ここには研究・開発したプログラムのソースコード等を書きましょう.

%%%%%%%%%%%%%%%%%%%%%%%%%%%%%%%%%%%%%%%%%%%%%%%%%%
\chapter*{謝辞}
ここに研究の謝辞.主にご協力いただいた方などを書きましょう.

%%%%%%%%%%%%%%%%%%%%%%%%%%%%%%%%%%%%%%%%%%%%%%%%%%
%%%%%%参考文献%%%%%%

\end{document}

%%%%%%document end%%%%%%

